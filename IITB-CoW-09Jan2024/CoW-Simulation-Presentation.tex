\documentclass[10pt]{beamer}
%\usepackage[utf8]{inputenc}
\usepackage{multirow,rotating}
%\usepackage{color}
%\usepackage{hyperref}
\usepackage{tikz-cd}
\usepackage{array}
\usepackage{siunitx}
\usepackage{mathtools,nccmath}%
\usepackage{
	%etoolbox, 
	xparse} 
\usetheme{CambridgeUS}
\usecolortheme{dolphin}
\usepackage{longtable}
\usepackage{gensymb}
\usepackage{tabularray}
\UseTblrLibrary{booktabs}
\usepackage{amsmath}


% set colors
\definecolor{myNewColorA}{RGB}{158, 27,50}
\definecolor{myNewColorB}{RGB}{158, 27,50}
\definecolor{myNewColorC}{RGB}{158, 27,50} % {130,138,143}
\setbeamercolor*{palette primary}{bg=myNewColorC}
\setbeamercolor*{palette secondary}{bg=myNewColorB, fg = white}
\setbeamercolor*{palette tertiary}{bg=myNewColorA, fg = white}
\setbeamercolor*{titlelike}{fg=myNewColorA}
\setbeamercolor*{title}{bg=myNewColorA, fg = white}
\setbeamercolor*{item}{fg=myNewColorA}
\setbeamercolor*{caption name}{fg=myNewColorA}
\usefonttheme{professionalfonts}
%\usepackage{natbib}
\usepackage{hyperref}
%------------------------------------------------------------
%\titlegraphic{\includegraphics[height=0.75cm]{iitb1.png}} 

% logo of my university
\begin{filecontents*}[overwrite]{\jobname.bib}
	
	@article{YOON2007886,
		title = {Heat integration analysis for an industrial ethylbenzene plant using pinch analysis},
		journal = {Applied Thermal Engineering},
		volume = {27},
		number = {5},
		pages = {886-893},
		year = {2007},
		issn = {1359-4311},
		doi = {https://doi.org/10.1016/j.applthermaleng.2006.09.001},
		author = {Sung-Geun Yoon and Jeongseok Lee and Sunwon Park}
	}
	
\end{filecontents*}

\usepackage[style=verbose,backend=biber]{biblatex}
\addbibresource{\jobname.bib}

\titlegraphic{%
	\includegraphics[width=3.0cm]{iitb1.png}
}

\setbeamerfont{title}{size=\large}
\setbeamerfont{subtitle}{size=\small}
\setbeamerfont{author}{size=\small}
\setbeamerfont{date}{size=\footnotesize}
\setbeamerfont{institute}{size=\footnotesize}
\title[CL308]{Process Simulation using DWSIM}%title
%\subtitle{ }%%subtitle
%\author[Priyam Nayak]{Priyam Nayak - 214026014\inst{1}}%%authors
\author[Priyam Nayak]{Priyam Nayak, Ph.D. Scholar}
%\institute[IITB]{Indian Institute of Technology Bombay\inst{1}}
\institute[IITB]{Indian Institute of Technology Bombay}
\date[\textcolor{white}{DWSIM Simulation}]
{CL308 - Course on Wheels\\ DWSIM Simulation\\ Jan 09, 2024}

%------------------------------------------------------------
%This block of commands puts the table of contents at the 
%beginning of each section and highlights the current section:
%\AtBeginSection[]
%{
	%  \begin{frame}
		%    \frametitle{Contents}
		%    \tableofcontents[currentsection]
		%  \end{frame}
	%}
\AtBeginSection[]{
	\begin{frame}
		\vfill
		\centering
		\begin{beamercolorbox}[sep=8pt,center,shadow=true,rounded=true]{title}
			\usebeamerfont{title}\insertsectionhead\par%
		\end{beamercolorbox}
		\vfill
	\end{frame}
}
% ------Contents below------
%------------------------------------------------------------

\begin{document}
	
	%The next statement creates the title page.
	\frame{\titlepage}
	\begin{frame}
		\frametitle{Outline}
		\tableofcontents
	\end{frame}
	
	
	% consider removing it if it's too redundant
	\AtBeginSection[]
	{
		\begin{frame}
			\frametitle{Table of Contents}
			\tableofcontents[currentsection]
		\end{frame}
	}
	
	%------------------------------------------------------------
	\section{Binary Column Distillation}
	\begin{frame}{Problem Statement}
		A binary mixture containing 40 mol\% methanol and 60 mol\% isopropanol is to be distilled. The mixture consists of equimolar amounts of vapor and liquid at a pressure of 1.01325 bar and a flowrate of 100 kmol/h. The desired purities are 99\%  methanol and 99\% isopropanol. The column is operated at 1.4 times the minimum reflux ratio. The toal condenser pressure is 1.01325 bar and reboiler pressure is 1.01325 bar.
	\end{frame}

\begin{frame}{Input Data}
	\begin{itemize}
		\item Components: Methanol-Isopropanol
		\item Thermodynamic Property Package: NRTL
		\item Feed Molar Flow Rate: 100 kmol/hr
		\item Feed Pressure: 1.01325 bar
		\item Feed Vapor Phase Mol Fraction: 0.5
		\item Mole Fraction of Methanol: 0.4
		\item Mole Fraction of Isopropanol: 0.6
	\end{itemize}
\end{frame}

\begin{frame}{Column Input}
	\begin{itemize}
		\item Mole Fraction of LK(Methanol) in Bottoms: 0.01
		\item Mole Fraction of HK(Isopropanol) in Distillate: 0.01
		\item Condenser Type: Total
		\item Condenser Pressure: 1.01325 bar
		\item Reboiler Pressure: 1.01325 bar
		\item $\frac{R}{R_{min}}$ = 1.4
	\end{itemize}
\end{frame}


\begin{frame}{Shortcut Column Results}
	\begin{itemize}
		\item Reflux Ratio: 3.84262
		\item Minimum Reflux Ratio: 2.74473
		\item Actual Number of Stages: 22
		\item Feed Stage Location: 10
	\end{itemize}
\end{frame}


\begin{frame}{Distillation Column Input}
	\begin{itemize}
		\item Actual Number of Stages: 22
		\item Feed Stage Location: 10
		\item Reflux Ratio: 3.84262
		\item Bottoms Flow Rate: 60.204 kmol/h
	\end{itemize}
\end{frame}

\section{Multicomponent Distillation}
	\begin{frame}{Problem Statement}
		
			A ternary mixture containing 30 mol\% methanol, 30 mol\% isopropanol and 40 mol\% n-propanol is to be distilled. The mixture consists of equimolar amounts of vapor and liquid at a pressure of 1.01325 bar and a flowrate of 100 kmol/h. The feed is to be separated through a sequence of columns such that in the first column, the composition of light key in bottoms is 1\% (by mole) and composition of heavy key is 1\% (by mole) in distillate. In the second column, the composition of light key in bottoms is 1\% (by mole) and composition of heavy key is 1\% (by mole) in distillate. The column is operated at 1.4 times the minimum reflux ratio. The toal condenser pressure is 1.01325 bar and reboiler pressure is 1.01325 bar.
		
%	A 100 kg/hr feed consisting 50\% (by mole) Benzene, 30\% Toluene and 20\% p-Xylene enters a distillation column at 1.01325 bar and 80$^\circ$C. The feed is to be separated through a sequence of columns such that in the first column, the composition of light key in bottoms is 0.1\% (by mole) and composition of heavy key is 1\% (by mole) in distillate. In the second column, the composition of light key in bottoms is 1\% (by mole) and composition of heavy key is 1\% (by mole) in distillate. The columns are operated at 1.3 times the minimum reflux ratio. For both the columns, the toal condenser pressure is 1.01325 bar and reboiler pressure is 1.01325 bar. Using Peng-Robinson property package, simulate the sequence of distillation column.
	\end{frame}

\begin{frame}{Input Data}
	\begin{itemize}
		\item Components: Methanol-Isopropanol-n-Propanol
		\item Thermodynamic Property Package: NRTL
		\item Feed Mass Flow Rate: 100 kmol/hr
		\item Feed Pressure: 1.01325 bar
		\item Feed Vapor Phase Mol Fraction: 0.5
		\item Mole Fraction of Methanol: 0.3
		\item Mole Fraction of Isopropanol: 0.3
		\item Mole Fraction of n-Propanol: 0.4
	\end{itemize}
\end{frame}

\begin{frame}{Column Input}
	\begin{itemize}
		\item Mole Fraction of LK(Methanol) in Bottoms: 0.01
		\item Mole Fraction of HK(Isopropanol) in Distillate: 0.01
		\item Condenser Type: Total
		\item Condenser Pressure: 1.01325 bar
		\item Reboiler Pressure: 1.01325 bar
		\item $\frac{R}{R_{min}}$ = 1.4
	\end{itemize}
\end{frame}

\begin{frame}{Shortcut Column-I Results}
	\begin{itemize}
		\item Reflux Ratio: 4.36978
		\item Minimum Reflux Ratio: 3.12127
		\item Actual Number of Stages: 21
		\item Feed Stage Location: 10
	\end{itemize}
\end{frame}


\begin{frame}{Distillation Column-I Input}
	\begin{itemize}
		\item Actual Number of Stages: 21
		\item Feed Stage Location: 10
		\item Reflux Ratio: 4.36978
		\item Bottoms Flow Rate: 70.4077 kmol/h
	\end{itemize}
\end{frame}




\section{Conversion/Yield Reactor}
\begin{frame}{Problem Statement}
100 kg/h of ethyl benzene at 260 \degree C and 1.5 bar is decomposed to form styrene and hydrogen. Products are at 250 \degree C and 1.2 bar. Assume the reaction to be vapour phase and 80\% conversion of ethyl benzene takes place. Using Peng-Robinson model of thermodynamics, simulate the conversion reactor.

\vspace{3ex}

Repeat the above problem with conversion as function of temperature (where T is in K) provided as 
\begin{equation*}
	f(T) = 0.0425(T+248)
\end{equation*}
\end{frame}

\begin{frame}{Input Data}
	\begin{itemize}
		\item Components: Ethylbenzene-Styrene-Hydrogen
		\item Thermodynamic Property Package: Peng-Robinson
		\item Feed Mass Flow Rate: 100 kg/hr
		\item Feed Pressure: 1.52 bar
		\item Feed Temperature: 260$^\circ$C
		\item Mole Fraction of Ethylbenzene: 1
		\item Mole Fraction of Styrene: 0
		\item Mole Fraction of Hydrogen: 0
	\end{itemize}
\end{frame}

\begin{frame}{Reaction and Reactor Input}
	\begin{itemize}
		\item Reaction: Ethylbenzene $\rightarrow$ Styrene + Hydrogen
		\item $X_{ethylbenzene}$: 80
		\item Reactor Outlet Temperature: 250$^\circ$C
		\item Reactor Pressure Drop: 0.3 bar
	\end{itemize}
\end{frame}


\section{Kinetic Reactor}
\begin{frame}{Problem Statement}
	2000 kg/h of feed consisting of pure acetone at 100 \degree C and 2 bar enters a plug flow reactor (volume of 200 $m^3$ and 10 $m$ length) to decompose into ketene and methane. The reaction rate is 
	\begin{equation*}
		-r_A = kP_{acetone} \frac{kmol}{m^3.hr}
	\end{equation*}
	where $P_{acetone}$ is the partial pressure of the Acetone in $Pa$. The reaction is assumed to follow arrhenius rate law where the pre-exponential factor is equal to 0.916 $hr^{-1}$ and the activation energy is equal to
	45000 $\frac{kJ}{kmol}$. The reactor is operated at 150 \degree C. Assuming the reaction to be in vapor phase and following
	Peng-Robinson property package, simulate a PFR.
\end{frame}

\begin{frame}{Input Data}
	\begin{itemize}
		\item Components: Acetone-Ketene-Methane
		\item Thermodynamic Property Package: Peng-Robinson
		\item Feed Mass Flow Rate: 2000 kg/hr
		\item Feed Pressure: 2 bar
		\item Feed Temperature: 100$^\circ$C
		\item Mole Fraction of Acetone: 1
		\item Mole Fraction of Ketene: 0
		\item Mole Fraction of Methane: 0
	\end{itemize}
\end{frame}

\begin{frame}{Reaction and Reactor Input}
	\begin{itemize}
		\item Reaction: Acetone $\rightarrow$ Ketene + Methane
		\item Pre-exponential factor: 0.916 $hr^{-1}$
		\item Activation Energy: 45000 $\frac{kJ}{kmol}$
		\item Reactor Volume: 200 $m^3$
		\item Reactor Length: 10 m
		\item Reactor Outlet Temperature: 150$^\circ$C
	\end{itemize}
\end{frame}
 
	\section*{Acknowledgement}  
	\begin{frame}
		
		\textcolor{myNewColorA}{\huge{\centerline{Thank you!}}}
		\vspace*{0.5cm}
		
		%\textcolor{myNewColorA}{\Large{\centerline{E-mail: 214026014@iitb.ac.in}}}
		
	\end{frame}
	
\end{document}